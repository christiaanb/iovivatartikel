\documentclass{paper}

\usepackage[utf8x]{inputenc}
\newcommand\textlambda{\ensuremath{\lambda}}
\newcommand{\clash}{CλaSH}

\usepackage{url}

\usepackage[dutch]{babel}

\usepackage{color}
\usepackage{xcolor}
\usepackage{listings}

\usepackage{courier}

\usepackage{caption}
\DeclareCaptionFont{white}{\color{white}}
\DeclareCaptionFormat{listing}{\colorbox{gray}{\parbox{\textwidth}{#1#2#3}}}
\captionsetup[lstlisting]{format=listing,labelfont=white,textfont=white}

\lstset{language=Haskell,captionpos=t,tabsize=2,frame=none,keywordstyle=\color{blue},commentstyle=\color{darkgreen},stringstyle=\color{red},numbers=left,numberstyle=\tiny,numbersep=5pt,breaklines=true,showstringspaces=false,basicstyle=\footnotesize\ttfamily}

\begin{document}
\pagestyle{empty}
\section*{Functioneel Hardware ontwerpen in \clash}
\subsubsection*{Christiaan Baaij \& Matthijs Kooijman}

De ontwikkeling van processtechnologie voor microchips ging de afgelopen decennia velen malen sneller dan de ontwikkelingen op het gebied van hardware ontwerp methodologi\"{e}n.
Als gevolg hiervan kunnen we nu meer transistoren op een chip zetten dan dat we nuttig kunnen gebruiken.
Binnen de CAES vakgroep\footnote{\url{http://caes.ewi.utwente.nl/}} van de Universiteit Twente is er een hardwarebeschrijvingstaal ontwikkeld, \clash\footnote{\clash: CAES Language for Synchronous Hardware} genaamd, welke gebaseerd is op de functionele taal Haskell\footnote{\url{http://haskell.org}}.
Omdat \clash\ een functionele hardwarebeschrijvingstaal is kunnen er een stuk makkelijker generieke ontwerpen gemaakt worden.
Als gevolg hiervan worden grotere en complexere projecten behapbaar voor de hardwareontwerper --- zo kunnen al die transistoren eindelijk goed gebruikt worden.

\subsubsection*{Hardware en Functionele Talen}
Combinatorische schakeling, oftewel hardware, zijn evenvouding wiskundig (als functies) te beschrijven.
Net als een functie levert een schakeling altijd hetzelfde resultaat bij herhaalde input.
Daarom zou een ontwerptaal voor hardware goed moeten zijn in de compositie, en decompositie, van (wiskundige) functies.
Van functionele talen, zoals Haskell, is bekend dat deze talen inderdaad zeer goed zijn in de compositie en decompositie van functies; vandaar dat functionele talen en hardware dus zo goed bij elkaar passen.
Een bijkomend voordeel van functionele talen is dat functies zogenaamde ``First-Class citizens'' zijn, wat betekent dat functies zelf zowel een argument als resultaat van een functie kunnen zijn.
Deze functionaliteit maakt het binnen \clash\ mogelijk om heel generieke en parameteriseerbare hardware ontwerpen te maken.
Generieke ontwerpen zijn daarom in \clash\ veel natuurlijker te beschrijven dan in de traditionele hardwarebeschrijvingstalen, welke hun achtergrond hebben in imperatieve talen zoals C.

\subsubsection*{Basis hardware} % (fold)
\label{ssub:basis_hardware}

% subsubsection basis_hardware (end)

\subsubsection*{State} % (fold)
\label{ssub:state}

% subsubsection state (end)

\subsubsection*{Simple CPU} % (fold)
\label{ssub:simple_cpu}
Om te laten zien dat je met \clash\ meer kan dan alleen een paar simpile circuits bouwen demonstreren we het ontwerp van een simpele processor.
Dit voorbeeld zal gebruik maken van de verschillende abstractie-mechanismen die we tot onze beschikking hebben, zoals: hogere-orde functies, partial application, lambda-expressies, pattern matching, en type afleiding.
De processor zal bestaan uit 4 executie-eenheden, $fu_{0}\ldots{}fu_{3}$, (zie Figuur~\ref{img:highordcpu}) die elk een binaire operatie zullen uitvoeren.

\begin{figure}
\caption{CPU met 4 executie-eenheden}
\label{img:highordcpu}
\end{figure}

Elke executie-eenheid heeft 7 data inputs en twee adres inputs. De twee adressen geven aan welke van 7 data inputs als operanden van de functie gebruikt moeten worden.

Deze 7 data inputs bestaat uit: een externe input \emph{x}, twee constanten (0 en 1), en de oude uitgangswaarden van de 4 executie-eenheden. De output van de processor als geheel is de oude waar de van de executie-eenheid $fu_{3}$.

De executie-eenheden $fu_{1}$, $fu_{2}$, en $fu_{3}$ voeren een vooraf vastgestelde functie uit. Executie-eenheid $fu_{0}$ heeft daarintegen een extra \emph{opcode} ingang, zodat er uit verschillende functies gekozen kan worden.

\begin{lstlisting}[label=Simple_CPU,caption=Simple CPU,float=t]
type Word     = Signed D16
type CpuState = State (Vector D4 Word)

fu op inputs (a1, a2) =
  op (inputs!a1) (inputs!a2)

fu1 = fu (+)
fu2 = fu (-)
fu3 = fu (*)

data Opcode = ShiftL | Xor | Equal

multiop ShiftL = shiftL
multiop Xor    = xor
multiop Equal  = \a b -> if a == b then 1 else 0

fu0 c = fu (multiop c)

cpuA :: CpuState
  -> (Word, Opcode, Vector D4 (Index D7, Index D7))
  -> (CpuState, Word)
cpuA (State s) (x,opc,addrs) = (State s', out)
  where
    inputs  = [x,0,1] ++ s
    s'      = [ fu0 opc inputs (addrs!0)
              , fu1     inputs (addrs!1)
              , fu2     inputs (addrs!2)
              , fu3     inputs (addrs!3)
              ]
    out     = vlast s
\end{lstlisting}
% subsubsection simple_cpu (end)

\subsubsection*{Digitaal Filter} % (fold)
\label{ssub:digitaal_filter}

% subsubsection digitaal_filter (end)

\end{document}
